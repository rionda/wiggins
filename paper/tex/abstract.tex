\begin{abstract}
	%\todo[Matteo]{Shorten to fit in single column, with Categories and
	%Keywords.}
	Detecting information spreading across a network can be done by monitoring
	its nodes and has many practical applications: discovering new webpages,
	analyzing influence properties in network, and detecting failure propagation
	in electronic circuits or infections in public drinkable water systems. In
	practice, it is infeasible for anyone but the owner of the network (if
	existent) and government agencies to monitor all nodes at all times. In this
	work we study the constrained setting when the observer can only probe a
	small set of nodes at each time step to check whether new pieces of
	information (items) have reached those nodes.

	We formally define the problem through an infinite time generating process
	that places new items in subsets of nodes according to an unknown
	probability distribution. Items have an exponentially decaying
	\emph{freshness}, modeling their decreasing value. The observer uses a
	\emph{probing schedule} (i.e., a probability distribution over the set of
	nodes) to choose, at each time step, a small set of nodes to check for new
	items. The goal is to compute a schedule that minimizes the average
	freshness of undetected items. We present an algorithm, \algoname, to
	compute the optimal schedule through convex optimization, and then show how
	it can be adapted when the parameters of the problem must be learned or
	change over time. We also present a scalable variant of \algoname for the
	MapReduce framework. The results of our experimental evaluation on real
	social networks demonstrate the practicality of our approach.
\end{abstract}
