\section{Problem Definition}\label{sec:prelims}
In this section we introduce the \probname and formally define our goal. We
start by giving some intuition about \probname, and then introduce it formally.

In a social network, news and pictures are shared and re-shared over time, and
each such shared ``item'' reaches a certain set of users. A resource-limited
agent is interested in knowing what items get shared on the network. Also, the
agent would like to know as soon as possible that an item has been shared,
because the value of this piece of information decreases over time. Being
resource-limited, the only way that the agent can observe (catch) the sharing of
items is to probe some nodes (i.e., users) of the social network, for example by
accessing their profile and getting a list of the items that they received from
other users or that they shared. Therefore, the agent needs a probing schedule
that allows her to catch as many items as possible while minimizing the time
between the sharing of an item and its being caught. Equivalently, the schedule
must be such that the ``value'' of the information that is not caught by the
agent is minimized. Intuitively, the schedule must be such that nodes that
receive a high number of items shortly after they have been generated should
have a higher probability of being probed. These nodes act, in some sense, as
``information hubs''. As such, identifying them can be seen as the complement of
the \emph{influence maximization problem}~\citep{Kempe2003,Kempe2005}. In the
influence maximization problem we look for a set of nodes that \emph{generate}
information that reach most nodes. In the information hubs problem we are
interested in a set of nodes that \emph{receive} the most of information, thus
the most informative nodes for an observer.

The \probname involves studying a \emph{generating process} $\sys$, which
describes how items are generated (formalized in Def.~\ref{def:generating}).
This is an unknown process that the agent observes by probing nodes. The
\probname has two parameters:
\begin{itemize*}
	\item a \emph{probe limit} parameter $c\in\mathbb{N}$ denoting how many
		nodes the agent can probe at each time step. This parameter describes
		her ``resource limitations'', as she can not probe all the network at
		each time step.
	\item A \emph{freshness parameter} $\theta\in(0,1]$, which specifies
		how fast the (agent-perceived) value of items decreases over time.
\end{itemize*}
Given two fixed values for $\theta$ and $c$, the $(\theta,c)$-PHSP on $\sys$
requires to study $\sys$ as it will be formalized in Def.~\ref{def:phsp}.

We now move to introduce the \probname and its components more formally.

\begin{definition}[Generating Process]\label{def:generating}
	Let $G=(V,E)$ be a graph, $\F$ be a family of subsets of $V$, and $\pi$ a
	function from $\F$ to $[0,1]$. A \emph{generating process} $\sys =
	(V,\F,\pi)$ is an \emph{infinite time process} that at each time $t$,
	generates a collection $\S_t\subseteq\F$ of subsets of $V$ such that $S\in
	\F$ is included in $\S_t$ with probability $\pi(S)$, independent of other
	sets generated at $t$ and at time $t'<t$.
\end{definition}

For example, let $\sys$ represents the process of generation and diffusion of
rumors in the graph $G$. At each time $t$, a set of rumors is generated, one
rumor for each set $S\in\S_t$ and it \emph{instantaneously} spreads to all
nodes in $S$, and then \emph{stops spreading}. This is a natural model of rumor
spreading as observed by the \emph{resource-limited} agent that can not monitor
all the network continuously at all times: the diffusion of a rumor happens too
fast for its progress to be observable by the resource-limited agent, to whom
the rumor appears therefore to spread instantaneously.

As we mentioned, the agent can actually observe new rumors only by
\emph{probing} up to $c$ nodes at each time $t$, checking whether any new
rumor reached them. We can formalize this idea as follows.

An \emph{item} is a pair $(t,S)$, for any $t$ and any $S\in\S_t$. For any
node $v\in V$, let $I_{t,v}$ be the set of items $(t',S)$ such that ``reached''
$v$ up to and including time $t$:
\[
	I_{t,v}=\set{(t',S) ~:~ t'\le t \mbox{ and } v\in S}\enspace.
\]
By \emph{probing} a node $v\in V$ at time $u$ we mean observing and
collecting the set $I_{t,}$. Probing an element is not (assumed to be) a
computationally expensive operation: accessing (probing) an element gives
immediate access to $I_{t,u}$. This is a realistic assumption: for example,
accessing the profile of a user in a social network gives access to the list of
items that reached that user.

An item $(t',S)$ is \emph{caught at time $t\geq t'$} iff:
\begin{enumerate*}
	\item a node $v\in S$ is probed at time $t$; and
	\item no node in $S$ was probed in the interval $[t',t-1]$.
\end{enumerate*}

The larger the interval between the time an item is generated and the time it is
caught, the less valuable that item is for the agent. This idea is formalized
by the \emph{$\theta$-freshness} of an item.

\begin{definition}[$\theta$-freshness]
	Let $\theta$ be the freshness parameter of a $(\theta,c)$-PHSP on $\sys$.
	The \emph{$\theta$-freshness} of an uncaught item $(t,S)$ at time $t'\ge t$,
	is $\theta^{t'-t}$.
\end{definition}
We believe that the choice of an exponential decaying freshness is realistic, as
the value of information decreases fast with time. Moreover it allows for a
rigorous theoretical study of \probname. Other definitions of freshness are
possible and we plan to study them in the future.

At each time step $t$, the resource-limited agent picks $c$ nodes to probe
according to a probing schedule, defined as follows.

\begin{definition}[$c$-Schedule]\label{def:schedule}
	Let $c$ be the probing limit parameter of a $(\theta,c)$-PHSP on $\sys$, and
	let $\p$ be a probability distribution over the set $V$ of nodes. A
	\emph{(probing) $c-$schedule} $\sigma=(c,\p)$ is an infinite time process
	that at each time $t$ generates a set $P_t\subseteq V$ of size $|P_t|\le c$
	such that the elements of $P_t$ are chosen independently according to $\p$,
	independently from $t$ and from the elements generated at previous time
	steps.
\end{definition}
At each time step, the agent probes the node in $P_t$. According to
Def.~\label{def:schedule}, a $c$-schedule must be \emph{memoryless}. We postpone
the study of adaptive or sequential schedules to future work.

The goal, for the resource-limited agent, is to design a probing $c$-schedule
catch items that have higher freshness (which correspond to those generated
most recently). In order to compare different schedules, we define a \emph{cost
measure for schedules} based on the sum of the freshness of \emph{uncaught}
items, as follows.

Let $\theta$ be the freshness parameter of a $(\theta,c)$-PHSP on $\sys$. The
\emph{$\theta$-load $L_\sys(t)$ of the generating process $\sys$ at time $t$} is
the sum of the $\theta$-freshness of \emph{uncaught} items at the end of time
step $t$ (i.e., after the agent has probed the nodes in the set $P_t$ generated
by a $c$-schedule.

\begin{definition}[$\theta$-cost]
	Given a $(\theta,c)$-PHSP on $\sys$, the $\theta$-\emph{cost} $\cost{c,p}$
	of a $c$-schedule $(c,\p)$ is the limit of the average expected
	$\theta$-load of $\sys$:
	\[
		\cost{c,p}=\lim\limits_{t\rightarrow\infty}\frac{1}{t}\sum_{t'=0}^{t}\E(L_\sys(\theta,t'))\enspace.
	\]
\end{definition}
The limit above always exists as proven in Lemma~\ref{lem:explicit}.
\todo{Is the proof of existence actually there?}

We have now all the ingredients to formally state the $(\theta,c)$-PHSP on a
generating process $\sys$.

\begin{definition}[\probname]\label{def:phsp}
	The goal in a $(\theta,c)$-{\probname}, or $(\theta,c)$-PHSP, for a
	generating process $\sys$ is to find a $c$-schedule $(c,\p)$ with the
	minimum $\theta$-cost.

	A $c$-schedule \emph{optimal} iff it has the minimum $\theta$-cost among all
	possible $c$-schedules.
\end{definition}

In $(\theta,c)$-PHSP, the parameter $\theta$ controls how fast the freshness of
an item decays, i.e., how fast they loose value for the agent. A value of
$\theta$ very close to $0$ requires to catch the item  as soon as they are
generated (or at most shortly thereafter). At the other extreme ($\theta=1$),
the items have always maximum freshness, and an ideal schedule should maximize
the number of caught items.  In other words, an ideal schedule $(c,\p)$ for
$(\theta, c)$-PHSP is such that elements of $U$ more likely to receive ``fresh''
items have higher probability of being probed (i.e., higher $\p$). Intuitively,
if we see the items as the ``information'' flowing in the network $G$, then an
ideal schedule should assign higher probability of being probed to nodes that
act as information hubs, as described at the beginning of this section. Hence by
examining the probability distribution $\p$ of an optimal schedule, we can
identify information hubs among the nodes.

In the following sections, we almost often drop the specification of the
parameters from $\theta$-freshness, $\theta$-cost, and $c$-schedule, and we
refer to them simply as ``freshness'', ``cost'', and schedule, because the
parameters $\theta$ and $c$ will be supposed to be fixed. Moreover, in
accordance to this modified use, we often denote $\p$ as a schedule, given that
the parameter $c$ is fixed, and use $\cost{\p}$ to denote the $\theta$-cost of
the $c$-schedule $(c,\p)$.
