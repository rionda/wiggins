\section{Model}\label{sec:model}




Our model consists of four components: %\vspace{-0.4cm}
\begin{enumerate}
\item 
an $n$-node directed network $G=(V,E)$, where $V=\{1,\dots,n\}$;
\item 
 a generating vector $\mathbf{\pi}=(\pi_1,\dots,\pi_n)$, such that the number of new items generated by node $i$ at each siep is a Poisson distribution with expectation $\pi_i$ (our methods apply also to the related model in which at each step each node $i$ generates one new item with probability $\pi_i$); 
\item
a propagation model $\mathbf{R}$ specifying how items are propagated in the network. For example, it can specify for each directed edge $v\rightarrow w$ in the network, the probability that a new item at node $v$ is propagated through this edge to node $w$

\ahmad{\item A degrading factor $\theta \in (0,1]$, such that the \ahmad{interest value} of an item after $t$ time steps of being undiscovered is $\theta^{t-1}$.}
\end{enumerate}

All generation and propagation events are independent. 
\st{We are interested here in propagation models in which most items reach only a subsets of the nodes in the network. }
\begin{definition}
The {\em informed set} of item $x$ is a set of nodes that includes the node that generated item $x$ and all the nodes to which it was propagated.
\end{definition} 



Our model defines a distribution  $\pi(S)$ on all possible informed sets $S\subseteq V$, where $\pi(S)$ is the probability that an item in the network is associated with an informed set $S$ i.e.
given that a new item was generated, $\pi(S)$ is the probability that its informed set is $S$.
 Since propagation time is much shorter than the discovery time we assume that an item is distributed immediately to all the nodes in its informed set.
%Our methods extend easily to related models, where: (1) each node $i$ receives items from its neighbors with probability $\phi'_i$; and (2) propagation of different edges are independent events.
%
%Suppose we have a group of $N$ people in a social network. In this network people follow each others' posts: a person $i$ posts a message (thought, news, etc) and all of his followers will see that message. We can represent such social network by a directed graph whose nodes are the $n$ individuals in the social network, and there is an edge $i \rightarrow j$ if $j$ follows $i$. \ahmad{We may alternate between person or node (that represents the person).}
%
%
%At each time step the person $i$ generates a \emph{wrong} message, we call \emph{rumor}, with probability $\pi_i$, and receives some rumors from the people he follows. Let $M(i,t)$ be the set of all these rumors (the newly generated and the received ones). He also reposts each rumor in $M(i,t-1)$ independently with probability $\rho$, and all of his followers receive that message.
%
%
%\ahmad{[{\bf move to last sec?}] We say a node (a person) has a rumor if he has posted or received the rumor, and by \emph{size} of a rumor $r$ we mean the number of nodes that have $r$.}
%At the end of each time step, a \emph{detector} can probe a node $i$ (by reading his page) and finds all the rumors that $i$ has, and by announcing the rumors via a \emph{broadcaster} all of these rumors will be removed in the whole network.  We assume a detector will probe \ahmad{a node at a moment between two consecutive time steps}. A \emph{deterministic/randomized schedule} is an algorithm for probing the nodes. 

The monitoring algorithm can probe one node at each step
and our goal is to construct a probing schedule that discovers at least one copy of each new item in the network as fast as possible.
While there can be complicated schedules that are hard to construct and implement, a major contribution of this work is the proof that simple, memoryless schedules, can achieve excellent performance.

% We consider both deterministic and randomized (memoryless) schedules.
%
%%Formally 
%\begin{definition}
%%\item
%%\begin{enumerate}
%%\item
%A \emph{deterministic $c$-schedule} is a function  
%%\begin{equation*}
%$$\S: \mathbb{N}\rightarrow \set{1,\ldots,N}^c,$$
%%\end{equation*}
%assigning $c$ nodes to be probed in each time step $t\in\mathbb{N}$. The schedule is 
%\emph{$\ell$-cyclic} if it repeats itself every $\ell$ steps.
%% \item \ahmad{A deterministic schedule is \emph{$\ell$-cyclic} if it repeats itself every $\ell$ steps.}
%%\item 
%\end{definition}
%
%An optimal, or close to optimal, deterministic schedule can have a long cycle and thus can be expensive to store and/or to execute. Instead, we show that simple to implement randomized, memoryless schedules provide an efficient alternative to the optimal deterministic schedule.
%

\begin{definition}
A \emph{randomized, memoryless schedule}, is a probability distribution $\p = (p_1,\ldots,p_N)$,
such that at each step the schedule probes a node chosen independently according to the distribution $\mathbf{p}$.
%\end{enumerate}
\end{definition}
For simplicity, throughout this work, by a schedule we mean a randomized memoryless schedule.
When the schedule probes node $i$ at time $t$, all items that were generated at that node or reached that node by time $t-1$ are discovered (each item is not discovered in at least one step). 

\begin{definition}
Suppose $\set{x_1, \ldots, x_m}$ is the set of undiscovered items at a time $t$. Also suppose $x_i$ is generated at time $t-t_i$.  The \emph{missing value} at time $t$ is 
$$\mi(t) = \sum_{i=1}^m \theta^{t_i-1}.$$
\end{definition}




\begin{definition}
The cost of a schedule $\p$ is \ahmad{(or average over the time steps...)}
\begin{align*}
\cost{\p} = \lim_{t\rightarrow \infty} \mi(t),
\end{align*}
A schedule $\p$ is stable for a given model if $\cost{\p}$ is bounded.
\end{definition}





\begin{definition}[Item-Catching Problem]
 Given a network $G$, the set of possible informed sets and their probabilities, find a schedule $\p^*$ that minimizes $\cost{\p}$.
 \end{definition}

Note that $\cost{\p}$ is unbounded \emph{only if} $\theta = 1$. However, the cost of the optimal schedule is always bounded. To see that, consider a uniform distribution schedule that probes each node with probability $1/n$. In this case, obviously for every set $S\subseteq V$ the fraction $\frac{1}{1-\theta(1-\p(S))}$ is defined (and bounded) and thus the cost the uniform schedule is bounded, ant thus, the cost of the optimal schedule is bounded.

Note that in practice we may not have access to the full set of possible informed sets, nor their corresponding probabilities. In the next section, we first see how we can compute \emph{a solution} that is close the optimal schedule, and then, we study how having a small sample from the possible informed sets can give us enough power to implement our method confidently. 

