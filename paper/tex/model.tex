\section{Problem Definition}\label{sec:prelims}
In this section we introduce the \probname and formally define our goal. We
define it in very general but formal terms and present examples to
show how it applies to information diffusion in networks.

\begin{definition}[Generating Process]
Let $U$ be a universe, $\F$ be a family of subsets of $U$, and
$\pi$ a function from $\F$ to $[0,1]$. A \emph{generating process} $\sys = (U,
\F,\pi)$ is an \emph{infinite time process} that at each time $t$, generates a
collection $\S_t\subseteq\F$ of subsets of $U$ such that $S\in \F$ is included
in $\S_t$ with probability $\pi(S)$, independent of other sets. 
\end{definition}

As an example, in a graph $G=(V,E)$, if we let $U=V$, then the process $\sys$
can, for specific choices of $\pi$ and $\F$, represent the process of
generation and diffusion of rumors in the graph. At each time $t$, a set of
rumors is generated, one rumor for each set $S\in\S_t$, and it instantaneously
diffuse to all nodes in $S$, after which it stops diffusing. This is a natural
model of rumors spreading as observed by a resource-limited agent that can not
monitor all the network continuously at all times: the diffusion of a rumor
happen too fast for its progress to be observable by such a resource-limited
agent, to whom the rumor appears therefore to diffuse instantaneously. 

Moreover, the agent can actually observe new rumors only by \emph{probing} a few
nodes at each time $t$, checking whether any new rumor reached them. Formally,
we can formulate this idea as follows.

An \emph{item} is a pair $(t,S)$, for any $t$ and any $S\in\S_t$. For any
element $u\in U$, let $I_{t,u}$ be the set of items $(t',S)$ such that $t'\le t$
and $u\in S$:
\[
	I_{t,u}=\set{(t',S) ~:~ t'\le t \mbox{ and } u\in S}\enspace.
\]
By \emph{probing} an element $u\in U$ at time $u$ we mean observing and
collecting the set $I_{t,u}$. Note that probing an element is not (assumed to
be) a computationally expensive operation: accessing (probing) an element gives
immediate access to $I_{t,u}$.

An item $(t',S)$ is \emph{caught at time $t\geq t'$} iff: \emph{1.}~an element
$u\in S$ is probed at time $t$; and \emph{2.} no element of $S$ was probed in
the interval $[t',t-1]$.

At each time step $t$, the resource-limited agent chooses a limited number
of nodes to probe, according to a probing schedule, defined as follows.

\begin{definition}[$c$-Schedule]
Let $\p$ be a probability distribution over the universe $U$, and $c$ be a
positive natural. A \emph{(probing) $c-$schedule} $\sigma=(c,\p)$ is a infinite
time process that at each time $t$ generates a set $P_t\subseteq U$ of size
$|P_t|\le c$ such that the elements of $P_t$ are chosen independently according
to $\p$, independently from $t$ and from the elements generated at previous time
steps.
\end{definition}

The goal, for the resource-limited agent, is to design a probing $c$-schedule
catch items that have been generated recently, as they are intuitively more
valuable. We measure the value of an item by its \emph{freshness}, which decays
exponentially with time, according to a agent-defined parameter
$\theta\in(0,1)$. 

\begin{definition}[$\theta$-freshness]
	Let $\theta\in(0,1)$. The \emph{freshness} of an uncaught item $(t,S)$ at
	time $t'\ge t$, is $\theta^{t'-t}$. 
\end{definition}

In order to compare different schedules, we define a \emph{cost measure for
schedules} based on the sum of the freshness of uncaught items, as follows.

The \emph{$\theta$-load $L_\sys(\theta, t)$ of the generating process $\sys$ at
time $t$} is the sum of the $\theta$-freshness of uncaught items at the end of
time step $t$ (i.e., after the agent has probed the elements of $U$ according to
a schedule). 

\begin{definition}[$\theta$-cost]
	The $\theta$-\emph{cost} $\cost(c,p)$ of a $c$-schedule $(c,\p)$ is the
	limit of the average expected $\theta$-load of $\sys$:
	\[
	\lim\limits_{t\rightarrow\infty}\frac{1}{t}\sum_{t'=0}^{t}\E(L_\sys(\theta,t'))\enspace.
	\]
\end{definition}
The limit above always exists as proven in Lemma~\ref{lem:explicit}.
\todo{Is the proof of existence actually there?}

The problem we focus on in this work is to find a schedule with minimum cost.

\begin{definition}[\probname]
 The goal in a $(\theta,c)$-{\probname}, or $(\theta,c)$-PHSP, for a generating
 system $\sys$ is to find  a $c$-schedule $(c,\p)$ with the minimum
 $\theta$-cost. 
 
 A $c$-schedule \emph{optimal} iff it has the minimum $\theta$-cost among all
 possible $c$-schedules.
\end{definition}

In $(\theta,c)$-PHSP, the parameter $\theta$ controls how fast the freshness of
an item decays, i.e., how fast they loose value for the agent. A value of
$\theta$ very close to $0$ requires to catch the item  as soon as they are
generated (or at most shortly thereafter). In the other extreme, the items have
always maximum freshness, and an ideal schedule should try to maximize the
number of caught items. 

In other words, an ideal schedule $(c,\p)$ for $(\theta, c)$-PHSP is such that
elements of $U$ more likely to receive ``fresh'' items have higher probability
of being probed (i.e., higher $\p$). Intuitively, if we see the items as the
``information'' flowing in the network $G$, then an ideal schedule should assign
higher probability of being probed to nodes that act as \emph{information hubs}.
Hence by examining the probability distribution $\p$ of an optimal schedule, we
can identify information hubs among the nodes. As such, the problem we deal with
in this work can be seens as the complement of the ``influence maximization
problem''~\citep{Kempe2003,Kempe2005}. In the influence maximization we look for
a set of nodes that generate the information that reach most nodes. In the
information hubs problem we are looking for the set of nodes that receive the
most amount of information, thus the most informative nodes.
